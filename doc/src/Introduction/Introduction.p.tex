%%
%% Automatically generated file from DocOnce source
%% (https://github.com/hplgit/doconce/)
%%
%%
% #ifdef PTEX2TEX_EXPLANATION
%%
%% The file follows the ptex2tex extended LaTeX format, see
%% ptex2tex: http://code.google.com/p/ptex2tex/
%%
%% Run
%%      ptex2tex myfile
%% or
%%      doconce ptex2tex myfile
%%
%% to turn myfile.p.tex into an ordinary LaTeX file myfile.tex.
%% (The ptex2tex program: http://code.google.com/p/ptex2tex)
%% Many preprocess options can be added to ptex2tex or doconce ptex2tex
%%
%%      ptex2tex -DMINTED myfile
%%      doconce ptex2tex myfile envir=minted
%%
%% ptex2tex will typeset code environments according to a global or local
%% .ptex2tex.cfg configure file. doconce ptex2tex will typeset code
%% according to options on the command line (just type doconce ptex2tex to
%% see examples). If doconce ptex2tex has envir=minted, it enables the
%% minted style without needing -DMINTED.
% #endif

% #define PREAMBLE

% #ifdef PREAMBLE
%-------------------- begin preamble ----------------------

\documentclass[%
oneside,                 % oneside: electronic viewing, twoside: printing
final,                   % draft: marks overfull hboxes, figures with paths
10pt]{article}

\listfiles               %  print all files needed to compile this document

\usepackage{relsize,makeidx,color,setspace,amsmath,amsfonts,amssymb}
\usepackage[table]{xcolor}
\usepackage{bm,ltablex,microtype}

\usepackage[pdftex]{graphicx}

\usepackage{ptex2tex}
% #ifdef MINTED
\usepackage{minted}
\usemintedstyle{default}
% #endif

\usepackage[T1]{fontenc}
%\usepackage[latin1]{inputenc}
\usepackage{ucs}
\usepackage[utf8x]{inputenc}

\usepackage{lmodern}         % Latin Modern fonts derived from Computer Modern

% Hyperlinks in PDF:
\definecolor{linkcolor}{rgb}{0,0,0.4}
\usepackage{hyperref}
\hypersetup{
    breaklinks=true,
    colorlinks=true,
    linkcolor=linkcolor,
    urlcolor=linkcolor,
    citecolor=black,
    filecolor=black,
    %filecolor=blue,
    pdfmenubar=true,
    pdftoolbar=true,
    bookmarksdepth=3   % Uncomment (and tweak) for PDF bookmarks with more levels than the TOC
    }
%\hyperbaseurl{}   % hyperlinks are relative to this root

\setcounter{tocdepth}{2}  % levels in table of contents

% newcommands for typesetting inline (doconce) comments
\newcommand{\shortinlinecomment}[3]{{\color{red}{\bf #1}: #2}}
\newcommand{\longinlinecomment}[3]{{\color{red}{\bf #1}: #2}}

% --- fancyhdr package for fancy headers ---
\usepackage{fancyhdr}
\fancyhf{} % sets both header and footer to nothing
\renewcommand{\headrulewidth}{0pt}
\fancyfoot[LE,RO]{\thepage}
% Ensure copyright on titlepage (article style) and chapter pages (book style)
\fancypagestyle{plain}{
  \fancyhf{}
  \fancyfoot[C]{{\footnotesize \copyright\ 1999-2019, Morten Hjorth-Jensen. Released under CC Attribution-NonCommercial 4.0 license}}
%  \renewcommand{\footrulewidth}{0mm}
  \renewcommand{\headrulewidth}{0mm}
}
% Ensure copyright on titlepages with \thispagestyle{empty}
\fancypagestyle{empty}{
  \fancyhf{}
  \fancyfoot[C]{{\footnotesize \copyright\ 1999-2019, Morten Hjorth-Jensen. Released under CC Attribution-NonCommercial 4.0 license}}
  \renewcommand{\footrulewidth}{0mm}
  \renewcommand{\headrulewidth}{0mm}
}

\pagestyle{fancy}


\usepackage[framemethod=TikZ]{mdframed}

% --- begin definitions of admonition environments ---

% --- end of definitions of admonition environments ---

% prevent orhpans and widows
\clubpenalty = 10000
\widowpenalty = 10000

% --- end of standard preamble for documents ---


% insert custom LaTeX commands...

\raggedbottom
\makeindex
\usepackage[totoc]{idxlayout}   % for index in the toc
\usepackage[nottoc]{tocbibind}  % for references/bibliography in the toc

%-------------------- end preamble ----------------------

\begin{document}

% matching end for #ifdef PREAMBLE
% #endif

\newcommand{\exercisesection}[1]{\subsection*{#1}}


% ------------------- main content ----------------------



% ----------------- title -------------------------

\thispagestyle{empty}

\begin{center}
{\LARGE\bf
\begin{spacing}{1.25}
Data Analysis and Machine Learning: Getting started, our first data and Machine Learning encounters
\end{spacing}
}
\end{center}

% ----------------- author(s) -------------------------

\begin{center}
{\bf Morten Hjorth-Jensen${}^{1, 2}$} \\ [0mm]
\end{center}

\begin{center}
% List of all institutions:
\centerline{{\small ${}^1$Department of Physics, University of Oslo}}
\centerline{{\small ${}^2$Department of Physics and Astronomy and National Superconducting Cyclotron Laboratory, Michigan State University}}
\end{center}
    
% ----------------- end author(s) -------------------------

% --- begin date ---
\begin{center}
Apr 21, 2019
\end{center}
% --- end date ---

\vspace{1cm}


\subsection{Introduction}

Our emphasis throughout this series of lectures  
is on understanding the mathematical aspects of
different algorithms used in the fields of data analysis and machine learning. 

However, where possible we will emphasize the
importance of using available software. We start thus with a hands-on
and top-down approach to machine learning. The aim is thus to start with
relevant data or data we have produced 
and use these to introduce statistical data analysis
concepts and machine learning algorithms before we delve into the
algorithms themselves. The examples we will use in the beginning, start with simple
polynomials with random noise added. We will use the Python
software package \href{{http://scikit-learn.org/stable/}}{Scikit-learn} and
introduce various machine learning algorithms to make fits of
the data and predictions. We move thereafter to more interesting
cases such as the simulation of financial transactions or disease
models. These are examples where we can easily set up the data and
then use machine learning algorithms included in for example
\textbf{scikit-learn}. 

These examples will serve us the purpose of getting
started. Furthermore, they allow us to catch more than two birds with
a stone. They will allow us to bring in some programming specific
topics and tools as well as showing the power of various Python 
packages for machine learning and statistical data analysis.  

Here, we will mainly focus on two
specific Python packages for Machine Learning, scikit-learn and
tensorflow (see below for links etc).  Moreover, the examples we
introduce will serve as inputs to many of our discussions later, as
well as allowing you to set up models and produce your own data and
get started with programming.





\subsection{Software and needed installations}

We will make extensive use of Python as programming language and its
myriad of available libraries.  You will find
Jupyter notebooks invaluable in your work.  You can run \textbf{R}
codes in the Jupyter/IPython notebooks, with the immediate benefit of
visualizing your data. You can also use compiled languages like C++,
Rust, Fortran etc if you prefer. The focus in these lectures will be
on Python.


If you have Python installed (we strongly recommend Python3) and you feel
pretty familiar with installing different packages, we recommend that
you install the following Python packages via \textbf{pip} as 

\begin{enumerate}
\item pip install numpy scipy matplotlib ipython scikit-learn mglearn sympy pandas pillow 
\end{enumerate}

\noindent
For Python3, replace \textbf{pip} with \textbf{pip3}.

For OSX users we recommend, after having installed Xcode, to
install \textbf{brew}. Brew allows for a seamless installation of additional
software via for example 

\begin{enumerate}
\item brew install python3
\end{enumerate}

\noindent
For Linux users, with its variety of distributions like for example the widely popular Ubuntu distribution,
you can use \textbf{pip} as well and simply install Python as 

\begin{enumerate}
\item sudo apt-get install python3  (or python for pyhton2.7)
\end{enumerate}

\noindent
etc etc. 



\subsection{Python installers}

If you don't want to perform these operations separately and venture
into the hassle of exploring how to set up dependencies and paths, we
recommend two widely used distrubutions which set up all relevant
dependencies for Python, namely 

\begin{itemize}
\item \href{{https://docs.anaconda.com/}}{Anaconda}, 
\end{itemize}

\noindent
which is an open source
distribution of the Python and R programming languages for large-scale
data processing, predictive analytics, and scientific computing, that
aims to simplify package management and deployment. Package versions
are managed by the package management system \textbf{conda}. 

\begin{itemize}
\item \href{{https://www.enthought.com/product/canopy/}}{Enthought canopy} 
\end{itemize}

\noindent
is a Python
distribution for scientific and analytic computing distribution and
analysis environment, available for free and under a commercial
license.

\subsection{Useful Python packages}
Here we list several useful Python packages you may wish to install (if you use anaconda many of these are already there)

\begin{itemize}
\item Numpy

\item Pandas

\item Xarray

\item Scipy

\item Matplotlib

\item autodiff

\item sympy

\item Scikit-learn

\item tensorflow and keras

\item pytorch
\end{itemize}

\noindent
\subsection{Installing R, C++, cython or Julia}

You will also find it convenient to utilize R. Although we will mainly
use Python during lectures and in various projects and exercises, we
provide a full R set of codes for the same examples. Those of you
already familiar with R should feel free to continue using R, keeping
however an eye on the parallel Python set ups. Similarly, if you are a
Python afecionado, feel free to explore R as well.  Jupyter/Ipython
notebook allows you to run \textbf{R} codes interactively in your
browser. The software library \textbf{R} is tuned to statistically analysis
and allows for an easy usage of the tools we will discuss in these
lectures.

To install \textbf{R} with Jupyter notebook 
\href{{https://mpacer.org/maths/r-kernel-for-ipython-notebook}}{follow the link here}




\subsection{Installing R, C++, cython, Numba etc}


For the C++ aficionados, Jupyter/IPython notebook allows you also to
install C++ and run codes written in this language interactively in
the browser. Since we will emphasize writing many of the algorithms
yourself, you can thus opt for either Python or C++ (or Fortran or other compiled languages) as programming
languages.

To add more entropy, \textbf{cython} can also be used when running your
notebooks. It means that Python with the Jupyter/IPython notebook
setup allows you to integrate widely popular softwares and tools for
scientific computing. Similarly, the 
\href{{https://numba.pydata.org/}}{Numba Python package} delivers increased performance
capabilities with minimal rewrites of your codes.  With its
versatility, including symbolic operations, Python offers a unique
computational environment. Your Jupyter/IPython notebook can easily be
converted into a nicely rendered \textbf{PDF} file or a Latex file for
further processing. For example, convert to latex as 

\bccq
pycod jupyter nbconvert filename.ipynb --to latex 
\eccq

And to add more versatility, the Python package \href{{http://www.sympy.org/en/index.html}}{SymPy} is a Python library for symbolic mathematics. It aims to become a full-featured computer algebra system (CAS)  and is entirely written in Python. 

Finally, if you wish to use the light mark-up language 
\href{{https://github.com/hplgit/doconce}}{doconce} you can convert a standard ascii text file into various HTML 
formats, ipython notebooks, latex files, pdf files etc with minimal edits.



\subsection{Numpy examples and Important Matrix and vector handling packages}

There are several central software packages for linear algebra and eigenvalue problems. Several of the more
popular ones have been wrapped into ofter software packages like those from the widely used text \textbf{Numerical Recipes}. The original source codes in many of the available packages are often taken from the widely used
software package LAPACK, which follows two other popular packages
developed in the 1970s, namely EISPACK and LINPACK.  We describe them shortly here.

\begin{itemize}
  \item LINPACK: package for linear equations and least square problems.

  \item LAPACK:package for solving symmetric, unsymmetric and generalized eigenvalue problems. From LAPACK's website \href{{http://www.netlib.org}}{\nolinkurl{http://www.netlib.org}} it is possible to download for free all source codes from this library. Both C/C++ and Fortran versions are available.

  \item BLAS (I, II and III): (Basic Linear Algebra Subprograms) are routines that provide standard building blocks for performing basic vector and matrix operations. Blas I is vector operations, II vector-matrix operations and III matrix-matrix operations. Highly parallelized and efficient codes, all available for download from \href{{http://www.netlib.org}}{\nolinkurl{http://www.netlib.org}}.
\end{itemize}

\noindent
When dealing with matrices and vectors a central issue is memory
handling and allocation. If our code is written in Python the way we
declare these objects and the way they are handled, interpreted and
used by say a linear algebra library, requires codes that interface
our Python program with such libraries. For Python programmers,
\textbf{Numpy} is by now the standard Python package for numerical arrays in
Python as well as the source of functions which act on these
arrays. These functions span from eigenvalue solvers to functions that
compute the mean value, variance or the covariance matrix. If you are
not familiar with how arrays are handled in say Python or compiled
languages like C++ and Fortran, the sections in this chapter may be
useful. For C++ programmer, \textbf{Armadillo} is widely used library for
linear algebra and eigenvalue problems. In addition it offers a
convenient way to handle and organize arrays. We discuss this library
as well.   Before we proceed we believe  it may be convenient to repeat some basic features of 
 matrices and vectors.


\subsection{Basic Matrix Features}


% --- begin paragraph admon ---
\paragraph{Matrix properties reminder.}
\[
 \mathbf{A} =
      \begin{bmatrix} a_{11} & a_{12} & a_{13} & a_{14} \\
                                 a_{21} & a_{22} & a_{23} & a_{24} \\
                                   a_{31} & a_{32} & a_{33} & a_{34} \\
                                  a_{41} & a_{42} & a_{43} & a_{44}
             \end{bmatrix}\qquad
\mathbf{I} =
      \begin{bmatrix} 1 & 0 & 0 & 0 \\
                                 0 & 1 & 0 & 0 \\
                                 0 & 0 & 1 & 0 \\
                                 0 & 0 & 0 & 1
             \end{bmatrix}
\]
% --- end paragraph admon ---



\subsection{Basic Matrix Features}

% --- begin paragraph admon ---
\paragraph{}

The inverse of a matrix is defined by

\[
\mathbf{A}^{-1} \cdot \mathbf{A} = I
\]
% --- end paragraph admon ---




\subsection{Basic Matrix Features}


% --- begin paragraph admon ---
\paragraph{Matrix Properties Reminder.}


\begin{quote}
\begin{tabular}{ccc}
\hline
\multicolumn{1}{c}{ Relations } & \multicolumn{1}{c}{ Name } & \multicolumn{1}{c}{ matrix elements } \\
\hline
$A = A^{T}$                            & symmetric       & $a_{ij} = a_{ji}$                                                       \\
$A = \left (A^{T} \right )^{-1}$       & real orthogonal & $\sum_k a_{ik} a_{jk} = \sum_k a_{ki} a_{kj} = \delta_{ij}$             \\
$A = A^{ * }$                          & real matrix     & $a_{ij} = a_{ij}^{ * }$                                                 \\
$A = A^{\dagger}$                      & hermitian       & $a_{ij} = a_{ji}^{ * }$                                                 \\
$A = \left (A^{\dagger} \right )^{-1}$ & unitary         & $\sum_k a_{ik} a_{jk}^{ * } = \sum_k a_{ki}^{ * } a_{kj} = \delta_{ij}$ \\
\hline
\end{tabular}
\end{quote}

\noindent
% --- end paragraph admon ---




\subsection{Some famous Matrices}

\begin{itemize}
  \item Diagonal if $a_{ij}=0$ for $i\ne j$

  \item Upper triangular if $a_{ij}=0$ for $i > j$

  \item Lower triangular if $a_{ij}=0$ for $i < j$

  \item Upper Hessenberg if $a_{ij}=0$ for $i > j+1$

  \item Lower Hessenberg if $a_{ij}=0$ for $i < j+1$

  \item Tridiagonal if $a_{ij}=0$ for $|i -j| > 1$

  \item Lower banded with bandwidth $p$: $a_{ij}=0$ for $i > j+p$

  \item Upper banded with bandwidth $p$: $a_{ij}=0$ for $i < j+p$

  \item Banded, block upper triangular, block lower triangular....
\end{itemize}

\noindent
\subsection{Basic Matrix Features}


% --- begin paragraph admon ---
\paragraph{Some Equivalent Statements.}
For an $N\times N$ matrix  $\mathbf{A}$ the following properties are all equivalent

\begin{itemize}
  \item If the inverse of $\mathbf{A}$ exists, $\mathbf{A}$ is nonsingular.

  \item The equation $\mathbf{Ax}=0$ implies $\mathbf{x}=0$.

  \item The rows of $\mathbf{A}$ form a basis of $R^N$.

  \item The columns of $\mathbf{A}$ form a basis of $R^N$.

  \item $\mathbf{A}$ is a product of elementary matrices.

  \item $0$ is not eigenvalue of $\mathbf{A}$.
\end{itemize}

\noindent
% --- end paragraph admon ---




\subsection{Numpy and arrays}
\href{{http://www.numpy.org/}}{Numpy} provides an easy way to handle arrays in Python. The standard way to import this library is as
\bpycod
import numpy as np
\epycod
Here follows a simple example where we set up an array of ten elements, all determined by random numbers drawn according to the normal distribution,
\bpycod
n = 10
x = np.random.normal(size=n)
print(x)
\epycod
We defined a vector $x$ with $n=10$ elements with its values given by the Normal distribution $N(0,1)$.
Another alternative is to declare a vector as follows
\bpycod
import numpy as np
x = np.array([1, 2, 3])
print(x)
\epycod
Here we have defined a vector with three elements, with $x_0=1$, $x_1=2$ and $x_2=3$. Note that both Python and C++
start numbering array elements from $0$ and on. This means that a vector with $n$ elements has a sequence of entities $x_0, x_1, x_2, \dots, x_{n-1}$. We could also let (recommended) Numpy to compute the logarithms of a specific array as
\bpycod
import numpy as np
x = np.log(np.array([4, 7, 8]))
print(x)
\epycod

Here we have used Numpy's unary function $np.log$. This function is
highly tuned to compute array elements since the code is vectorized
and does not require looping. We normaly recommend that you use the
Numpy intrinsic functions instead of the corresponding \textbf{log} function
from Python's \textbf{math} module. The looping is done explicitely by the
\textbf{np.log} function. The alternative, and slower way to compute the
logarithms of a vector would be to write

\bpycod
import numpy as np
from math import log
x = np.array([4, 7, 8])
for i in range(0, len(x)):
    x[i] = log(x[i])
print(x)
\epycod
We note that our code is much longer already and we need to import the \textbf{log} function from the \textbf{math} module. 
The attentive reader will also notice that the output is $[1, 1, 2]$. Python interprets automagically our numbers as integers (like the \textbf{automatic} keyword in C++). To change this we could define our array elements to be double precision numbers as
\bpycod
import numpy as np
x = np.log(np.array([4, 7, 8], dtype = np.float64))
print(x)
\epycod
or simply write them as double precision numbers (Python uses 64 bits as default for floating point type variables), that is
\bpycod
import numpy as np
x = np.log(np.array([4.0, 7.0, 8.0])
print(x)
\epycod
To check the number of bytes (remember that one byte contains eight bits for double precision variables), you can use simple use the \textbf{itemsize} functionality (the array $x$ is actually an object which inherits the functionalities defined in Numpy) as 
\bpycod
import numpy as np
x = np.log(np.array([4.0, 7.0, 8.0])
print(x.itemsize)
\epycod


\subsection{Matrices in Python}

Having defined vectors, we are now ready to try out matrices. We can
define a $3 \times 3 $ real matrix $\hat{A}$ as (recall that we user
lowercase letters for vectors and uppercase letters for matrices)

\bpycod
import numpy as np
A = np.log(np.array([ [4.0, 7.0, 8.0], [3.0, 10.0, 11.0], [4.0, 5.0, 7.0] ]))
print(A)
\epycod
If we use the \textbf{shape} function we would get $(3, 3)$ as output, that is verifying that our matrix is a $3\times 3$ matrix. We can slice the matrix and print for example the first column (Python organized matrix elements in a row-major order, see below) as
\bpycod
import numpy as np
A = np.log(np.array([ [4.0, 7.0, 8.0], [3.0, 10.0, 11.0], [4.0, 5.0, 7.0] ]))
# print the first column, row-major order and elements start with 0
print(A[:,0]) 
\epycod
We can continue this was by printing out other columns or rows. The example here prints out the second column
\bpycod
import numpy as np
A = np.log(np.array([ [4.0, 7.0, 8.0], [3.0, 10.0, 11.0], [4.0, 5.0, 7.0] ]))
# print the first column, row-major order and elements start with 0
print(A[1,:]) 
\epycod
Numpy contains many other functionalities that allow us to slice, subdivide etc etc arrays. We strongly recommend that you look up the \href{{http://www.numpy.org/}}{Numpy website for more details}. Useful functions when defining a matrix are the \textbf{np.zeros} function which declares a matrix of a given dimension and sets all elements to zero
\bpycod
import numpy as np
n = 10
# define a matrix of dimension 10 x 10 and set all elements to zero
A = np.zeros( (n, n) )
print(A) 
\epycod
or initializing all elements to 
\bpycod
import numpy as np
n = 10
# define a matrix of dimension 10 x 10 and set all elements to one
A = np.ones( (n, n) )
print(A) 
\epycod
or as unitarily distributed random numbers (see the material on random number generators in the statistics part)
\bpycod
import numpy as np
n = 10
# define a matrix of dimension 10 x 10 and set all elements to random numbers with x \in [0, 1]
A = np.random.rand(n, n)
print(A) 
\epycod

As we will see throughout these lectures, there are several extremely useful functionalities in Numpy.
As an example, consider the discussion of the covariance matrix. Suppose we have defined three vectors
$\hat{x}, \hat{y}, \hat{z}$ with $n$ elements each. The covariance matrix is defined as 
\[
\hat{\Sigma} = \begin{bmatrix} \sigma_{xx} & \sigma_{xy} & \sigma_{xz} \\
                              \sigma_{yx} & \sigma_{yy} & \sigma_{yz} \\
                              \sigma_{zx} & \sigma_{zy} & \sigma_{zz} 
             \end{bmatrix},
\]
where for example
\[
\sigma_{xy} =\frac{1}{n} \sum_{i=0}^{n-1}(x_i- \overline{x})(y_i- \overline{y}).
\]
The Numpy function \textbf{np.cov} calculates the covariance elements using the factor $1/(n-1)$ instead of $1/n$ since it assumes we do not have the exact mean values. For a more in-depth discussion of the covariance and covariance matrix and its meaning, we refer you to the lectures on statistics. 
The following simple function uses the \textbf{np.vstack} function which takes each vector of dimension $1\times n$ and produces a $ 3\times n$ matrix $\hat{W}$
\[
\hat{W} = \begin{bmatrix} x_0 & y_0 & z_0 \\
                          x_1 & y_1 & z_1 \\
                          x_2 & y_2 & z_2 \\
                          \dots & \dots & \dots \\
                          x_{n-2} & y_{n-2} & z_{n-2} \\
                          x_{n-1} & y_{n-1} & z_{n-1}
             \end{bmatrix},
\]

which in turn is converted into into the $3 times 3$ covariance matrix
$\hat{\Sigma}$ via the Numpy function \textbf{np.cov()}. In our review of
statistical functions and quantities we will discuss more about the
meaning of the covariance matrix. Here we note that we can calculate
the mean value of each set of samples $\hat{x}$ etc using the Numpy
function \textbf{np.mean(x)}. We can also extract the eigenvalues of the
covariance matrix through the \textbf{np.linalg.eig()} function.

\bpycod
# Importing various packages
import numpy as np

n = 100
x = np.random.normal(size=n)
print(np.mean(x))
y = 4+3*x+np.random.normal(size=n)
print(np.mean(y))
z = x**3+np.random.normal(size=n)
print(np.mean(z))
W = np.vstack((x, y, z))
Sigma = np.cov(W)
print(Sigma)
Eigvals, Eigvecs = np.linalg.eig(Sigma)
print(Eigvals)
\epycod


\bpycod
import numpy as np
import matplotlib.pyplot as plt
from scipy import sparse
eye = np.eye(4)
print(eye)
sparse_mtx = sparse.csr_matrix(eye)
print(sparse_mtx)
x = np.linspace(-10,10,100)
y = np.sin(x)
plt.plot(x,y,marker='x')
plt.show()
\epycod


\subsection{Meet the Pandas}

Another useful Python package is
\href{{https://pandas.pydata.org/}}{pandas}, which is an open source library
providing high-performance, easy-to-use data structures and data
analysis tools for Python. The following simple example shows how we can, in an easy way make tables of our data. Here we define a data set which includes names, city of residence and age, and displays the data in an easy to read way. We will see repeated use of \textbf{pandas}, in particular in connection with classification of data. 

\bpycod
import pandas as pd
from IPython.display import display
data = {'Name': ["John", "Anna", "Peter", "Linda"], 'Location': ["Nairobi", "Napoli", "London", "Buenos Aires"], 'Age':[51, 21, 34, 45]}
data_pandas = pd.DataFrame(data)
display(data_pandas)
\epycod

Here are other examples where we use the \textbf{DataFrame} functionality to handle arrays
\bpycod
import numpy as np
import pandas as pd
from IPython.display import display
np.random.seed(100)
# setting up a 9 x 4 matrix
rows = 9
cols = 4
a = np.random.randn(rows,cols)
df = pd.DataFrame(a)
display(df)
print(df.mean())
print(df.std())
display(df**2)


df.columns = ['First', 'Second', 'Third', 'Fourth']
df.index = np.arange(9)

display(df)
print(df['Second'].mean() )
print(df.info())
print(df.describe())

from pylab import plt, mpl
plt.style.use('seaborn')
mpl.rcParams['font.family'] = 'serif'

df.cumsum().plot(lw=2.0, figsize=(10,6))
#plt.show()


df.plot.bar(figsize=(10,6), rot=15)
#plt.show()


b = np.arange(16).reshape((4,4))
print(b)

df1 = pd.DataFrame(b)

\epycod


\subsection{Reading Data and fitting Nuclear Masses}

In order to study various Machine Learning algorithms, we need to
access data. Acccessing data is an essential step in all machine
learning algorithms. In particular, setting up the so-called \textbf{design
matrix} (to be defined below) is often the first element we need in
order to perform our calculations. To set up the design matrix means
reading (and later, when the calculations are done, writing) data
in various formats, The formats span from reading files from disk,
loading data from databases and interacting with online sources
like web application programming interfaces (APIs).

In handling various input formats, we will mainly stay with \textbf{pandas},
a Python package which allows us, in a seamless and painless way, to
deal with a multitude of formats, from standard \textbf{csv} (comma separated
values) files, via \textbf{excel}, \textbf{html} to \textbf{hdf5} formats.  With \textbf{pandas}
and the \textbf{DataFrame}  and \textbf{Series} functionalities we are able to convert text data
into the calculational formats we need for a specific algorithm. And our code is going to be 
pretty close the basic mathematical expressions.

We are going to start with a classic case from nuclear physics, namely all
available data on binding energies. We will then show some of the
strength of packages like \textbf{scikit-learn} in fitting binding energies to
specific functions using linear regression first. Then, as a teaser, we will show you how 
you can easily implement other algorithms like decision trees and random forests and neural networks.

But before we really start with nuclear data, let's just look at some simpler polynomial fitting cases, such as,
(don't be offended) fitting straight lines!


\paragraph{Simple linear regression model using \textbf{scikit-learn}.}
We start with perhaps our simplest possible example, using \textbf{scikit-learn} to perform linear regression analysis on a data set produced by us. 

What follows is a simple Python code where we have defined a function
$y$ in terms of the variable $x$. Both are defined as vectors with  $100$ entries. 
The numbers in the vector $\hat{x}$ are given
by random numbers generated with a uniform distribution with entries
$x_i \in [0,1]$ (more about probability distribution functions
later). These values are then used to define a function $y(x)$
(tabulated again as a vector) with a linear dependence on $x$ plus a
random noise added via the normal distribution.


The Numpy functions are imported used the \textbf{import numpy as np}
statement and the random number generator for the uniform distribution
is called using the function \textbf{np.random.rand()}, where we specificy
that we want $100$ random variables.  Using Numpy we define
automatically an array with the specified number of elements, $100$ in
our case.  With the Numpy function \textbf{randn()} we can compute random
numbers with the normal distribution (mean value $\mu$ equal to zero and
variance $\sigma^2$ set to one) and produce the values of $y$ assuming a linear
dependence as function of $x$

\[
y = 2x+N(0,1),
\]

where $N(0,1)$ represents random numbers generated by the normal
distribution.  From \textbf{scikit-learn} we import then the
\textbf{LinearRegression} functionality and make a prediction $\tilde{y} =
\alpha + \beta x$ using the function \textbf{fit(x,y)}. We call the set of
data $(\hat{x},\hat{y})$ for our training data. The Python package
\textbf{scikit-learn} has also a functionality which extracts the above
fitting parameters $\alpha$ and $\beta$ (see below). Later we will
distinguish between training data and test data.

For plotting we use the Python package
\href{{https://matplotlib.org/}}{matplotlib} which produces publication
quality figures. Feel free to explore the extensive
\href{{https://matplotlib.org/gallery/index.html}}{gallery} of examples. In
this example we plot our original values of $x$ and $y$ as well as the
prediction \textbf{ypredict} ($\tilde{y}$), which attempts at fitting our
data with a straight line.

The Python code follows here.
\bpycod
# Importing various packages
import numpy as np
import matplotlib.pyplot as plt
from sklearn.linear_model import LinearRegression

x = np.random.rand(100,1)
y = 2*x+np.random.randn(100,1)
linreg = LinearRegression()
linreg.fit(x,y)
xnew = np.array([[0],[1]])
ypredict = linreg.predict(xnew)

plt.plot(xnew, ypredict, "r-")
plt.plot(x, y ,'ro')
plt.axis([0,1.0,0, 5.0])
plt.xlabel(r'$x$')
plt.ylabel(r'$y$')
plt.title(r'Simple Linear Regression')
plt.show()
\epycod

This example serves several aims. It allows us to demonstrate several
aspects of data analysis and later machine learning algorithms. The
immediate visualization shows that our linear fit is not
impressive. It goes through the data points, but there are many
outliers which are not reproduced by our linear regression.  We could
now play around with this small program and change for example the
factor in front of $x$ and the normal distribution.  Try to change the
function $y$ to

\[
y = 10x+0.01 \times N(0,1),
\]

where $x$ is defined as before.  Does the fit look better? Indeed, by
reducing the role of the noise given by normal distribution we see immediately that
our linear prediction seemingly reproduces better the training
set. However, this testing 'by the eye' is obviouly not satisfactory in the
long run. Here we have only defined the training data and our model, and 
have not discussed a more rigorous approach to the \textbf{cost} function.

We need more rigorous criteria in defining whether we have succeeded or
not in modeling our training data.  You will be surprised to see that
many scientists seldomly venture beyond this 'by the eye' approach. A
standard approach for the \emph{cost} function is the so-called $\chi^2$
function 

\[ \chi^2 = \frac{1}{n}
\sum_{i=0}^{n-1}\frac{(y_i-\tilde{y}_i)^2}{\sigma_i^2}, 
\] 

where $\sigma_i^2$ is the variance (to be defined later) of the entry
$y_i$.  We may not know the explicit value of $\sigma_i^2$, it serves
however the aim of scaling the equations and make the cost function
dimensionless.  

Minimizing the cost function is a central aspect of
our discussions to come. Finding its minima as function of the model
parameters ($\alpha$ and $\beta$ in our case) will be a recurring
theme in these series of lectures. Essentially all machine learning
algorithms we will discuss center around the minimization of the
chosen cost function. This depends in turn on our specific
model for describing the data, a typical situation in supervised
learning. Automatizing the search for the minima of the cost function is a
central ingredient in all algorithms. Typical methods which are
employed are various variants of \textbf{gradient} methods. These will be
discussed in more detail later. Again, you'll be surprised to hear that
many practitioners minimize the above function ''by the eye', popularly dubbed as 
'chi by the eye'. That is, change a parameter and see (visually and numerically) that 
the  $\chi^2$ function becomes smaller. 

There are many ways to define the cost function. A simpler approach is to look at the relative difference between the training data and the predicted data, that is we define 
the relative error as

\[
\epsilon_{\mathrm{relative}}= \frac{\vert \hat{y} -\hat{\tilde{y}}\vert}{\vert \hat{y}\vert}.
\]
We can modify easily the above Python code and plot the relative error instead
\bpycod
import numpy as np
import matplotlib.pyplot as plt
from sklearn.linear_model import LinearRegression

x = np.random.rand(100,1)
y = 5*x+0.01*np.random.randn(100,1)
linreg = LinearRegression()
linreg.fit(x,y)
ypredict = linreg.predict(x)

plt.plot(x, np.abs(ypredict-y)/abs(y), "ro")
plt.axis([0,1.0,0.0, 0.5])
plt.xlabel(r'$x$')
plt.ylabel(r'$\epsilon_{\mathrm{relative}}$')
plt.title(r'Relative error')
plt.show()
\epycod

Depending on the parameter in front of the normal distribution, we may
have a small or larger relative error. Try to play around with
different training data sets and study (graphically) the value of the
relative error.

As mentioned above, \textbf{scikit-learn} has an impressive functionality.
We can for example extract the values of $\alpha$ and $\beta$ and
their error estimates, or the variance and standard deviation and many
other properties from the statistical data analysis. 

Here we show an
example of the functionality of scikit-learn.
\bpycod
import numpy as np 
import matplotlib.pyplot as plt 
from sklearn.linear_model import LinearRegression 
from sklearn.metrics import mean_squared_error, r2_score, mean_squared_log_error, mean_absolute_error

x = np.random.rand(100,1)
y = 2.0+ 5*x+0.5*np.random.randn(100,1)
linreg = LinearRegression()
linreg.fit(x,y)
ypredict = linreg.predict(x)
print('The intercept alpha: \n', linreg.intercept_)
print('Coefficient beta : \n', linreg.coef_)
# The mean squared error                               
print("Mean squared error: %.2f" % mean_squared_error(y, ypredict))
# Explained variance score: 1 is perfect prediction                                 
print('Variance score: %.2f' % r2_score(y, ypredict))
# Mean squared log error                                                        
print('Mean squared log error: %.2f' % mean_squared_log_error(y, ypredict) )
# Mean absolute error                                                           
print('Mean absolute error: %.2f' % mean_absolute_error(y, ypredict))
plt.plot(x, ypredict, "r-")
plt.plot(x, y ,'ro')
plt.axis([0.0,1.0,1.5, 7.0])
plt.xlabel(r'$x$')
plt.ylabel(r'$y$')
plt.title(r'Linear Regression fit ')
plt.show()

\epycod
The function \textbf{coef} gives us the parameter $\beta$ of our fit while \textbf{intercept} yields 
$\alpha$. Depending on the constant in front of the normal distribution, we get values near or far from $alpha =2$ and $\beta =5$. Try to play around with different parameters in front of the normal distribution. The function \textbf{meansquarederror} gives us the mean square error, a risk metric corresponding to the expected value of the squared (quadratic) error or loss defined as
\[ MSE(\hat{y},\hat{\tilde{y}}) = \frac{1}{n}
\sum_{i=0}^{n-1}(y_i-\tilde{y}_i)^2, 
\] 

The smaller the value, the better the fit. Ideally we would like to
have an MSE equal zero.  The attentive reader has probably recognized
this function as being similar to the $\chi^2$ function defined above.

The \textbf{r2score} function computes $R^2$, the coefficient of
determination. It provides a measure of how well future samples are
likely to be predicted by the model. Best possible score is 1.0 and it
can be negative (because the model can be arbitrarily worse). A
constant model that always predicts the expected value of $\hat{y}$,
disregarding the input features, would get a $R^2$ score of $0.0$.

If $\tilde{\hat{y}}_i$ is the predicted value of the $i-th$ sample and $y_i$ is the corresponding true value, then the score $R^2$ is defined as
\[
R^2(\hat{y}, \tilde{\hat{y}}) = 1 - \frac{\sum_{i=0}^{n - 1} (y_i - \tilde{y}_i)^2}{\sum_{i=0}^{n - 1} (y_i - \bar{y})^2},
\]
where we have defined the mean value  of $\hat{y}$ as
\[
\bar{y} =  \frac{1}{n} \sum_{i=0}^{n - 1} y_i.
\]
Another quantity taht we will meet again in our discussions of regression analysis is 
 the mean absolute error (MAE), a risk metric corresponding to the expected value of the absolute error loss or what we call the $l1$-norm loss. In our discussion above we presented the relative error.
The MAE is defined as follows
\[
\text{MAE}(\hat{y}, \hat{\tilde{y}}) = \frac{1}{n} \sum_{i=0}^{n-1} \left| y_i - \tilde{y}_i \right|.
\]
Finally we present the 
squared logarithmic (quadratic) error
\[
\text{MSLE}(\hat{y}, \hat{\tilde{y}}) = \frac{1}{n} \sum_{i=0}^{n - 1} (\log_e (1 + y_i) - \log_e (1 + \tilde{y}_i) )^2,
\]

where $\log_e (x)$ stands for the natural logarithm of $x$. This error
estimate is best to use when targets having exponential growth, such
as population counts, average sales of a commodity over a span of
years etc. 

We will discuss in more
detail these and other functions in the various lectures.  We conclude this part with another example. Instead of 
a linear $x$-dependence we study now a cubic polynomial and use the polynomial regression analysis tools of scikit-learn. 

\bpycod
import matplotlib.pyplot as plt
import numpy as np
import random
from sklearn.linear_model import Ridge
from sklearn.preprocessing import PolynomialFeatures
from sklearn.pipeline import make_pipeline
from sklearn.linear_model import LinearRegression

x=np.linspace(0.02,0.98,200)
noise = np.asarray(random.sample((range(200)),200))
y=x**3*noise
yn=x**3*100
poly3 = PolynomialFeatures(degree=3)
X = poly3.fit_transform(x[:,np.newaxis])
clf3 = LinearRegression()
clf3.fit(X,y)

Xplot=poly3.fit_transform(x[:,np.newaxis])
poly3_plot=plt.plot(x, clf3.predict(Xplot), label='Cubic Fit')
plt.plot(x,yn, color='red', label="True Cubic")
plt.scatter(x, y, label='Data', color='orange', s=15)
plt.legend()
plt.show()

def error(a):
    for i in y:
        err=(y-yn)/yn
    return abs(np.sum(err))/len(err)

print (error(y))
\epycod

Similarly, using \textbf{R}, we can perform similar studies. 





\paragraph{Brief reminder on masses and binding energies.}
Let us return to nuclear physics and remind ourselves briefly about some basic features about binding
energies.  A basic quantity which can be measured for the ground
states of nuclei is the atomic mass $M(N, Z)$ of the neutral atom with
atomic mass number $A$ and charge $Z$. The number of neutrons is $N$.

Atomic masses are usually tabulated in terms of the mass excess defined by
\[
\Delta M(N, Z) =  M(N, Z) - uA,
\]
where $u$ is the Atomic Mass Unit 
\[
u = M(^{12}\mathrm{C})/12 = 931.49386 \hspace{0.1cm} \mathrm{MeV}/c^2.
\]
The nucleon masses are
\[
m_p = 938.27203(8)\hspace{0.1cm} \mathrm{MeV}/c^2 = 1.00727646688(13)u,
\] 
and
\[
m_n = 939.56536(8)\hspace{0.1cm} \mathrm{MeV}/c^2 = 1.0086649156(6)u.
\]

In the \href{{http://nuclearmasses.org/resources_folder/Wang_2017_Chinese_Phys_C_41_030003.pdf}}{2016 mass evaluation of by W.J.Huang, G.Audi, M.Wang, F.G.Kondev, S.Naimi and X.Xu}
there are xxx nuclei measured with an
accuracy of xxx MeV or better, and xxx nuclei measured with an
accuracy of greater than xxx MeV. For heavy nuclei one observes
several chains of nuclei with a constant $N-Z$ value whose masses are
obtained from the energy released in $\alpha$-decay.


The nuclear binding energy is defined as the energy required to break
up a given nucleus into its constituent parts of $N$ neutrons and $Z$
protons. In terms of the atomic masses $M(N, Z)$ the binding energy is
defined by


\[
BE(N, Z) = ZM_H c^2 + Nm_n c^2 - M(N, Z)c^2 ,
\]
where $M_H$ is the mass of the hydrogen atom and $m_n$ is the mass of the neutron.
In terms of the mass excess the binding energy is given by
\[
BE(N, Z) = Z\Delta_H c^2 + N\Delta_n c^2 -\Delta(N, Z)c^2 ,
\]
where $\Delta_H c^2 = 7.2890$ MeV and $\Delta_n c^2 = 8.0713$ MeV.


A popular and physically intuitive model which can be used to parametrize 
the experimental binding energies as function of $A$, is the so-called 
\textbf{liquid drop model}. The ansatz is based on the following expression

\[ 
BE(N,Z) = a_1A-a_2A^{2/3}-a_3\frac{Z^2}{A^{1/3}}-a_4\frac{(N-Z)^2}{A},
\]

where $A$ stands for the number of nucleons and the $a_i$s are parameters which are determined by a fit 
to the experimental data.  




To arrive at the above expression we have assumed that we can make the following assumptions:

\begin{itemize}
 \item There is a volume term $a_1A$ proportional with the number of nucleons (the energy is also an extensive quantity). When an assembly of nucleons of the same size is packed together into the smallest volume, each interior nucleon has a certain number of other nucleons in contact with it. This contribution is proportional to the volume.

 \item There is a surface energy term $a_2A^{2/3}$. The assumption here is that a nucleon at the surface of a nucleus interacts with fewer other nucleons than one in the interior of the nucleus and hence its binding energy is less. This surface energy term takes that into account and is therefore negative and is proportional to the surface area.

 \item There is a Coulomb energy term $a_3\frac{Z^2}{A^{1/3}}$. The electric repulsion between each pair of protons in a nucleus yields less binding. 

 \item There is an asymmetry term $a_4\frac{(N-Z)^2}{A}$. This term is associated with the Pauli exclusion principle and reflects the fact that the proton-neutron interaction is more attractive on the average than the neutron-neutron and proton-proton interactions.
\end{itemize}

\noindent
We could also add a so-called pairing term, which is a correction term that
arises from the tendency of proton pairs and neutron pairs to
occur. An even number of particles is more stable than an odd number. 


\paragraph{Organizing our data.}
Let us start with reading and organizing our data. 
We start with the compilation of masses and binding energies from 2016.
After having downloaded this file to our own computer, we are now ready to read the file and start structuring our data.


We start with preparing folders for storing our calculations and the data file over masses and binding energies. We import also various modules that we will find useful in order to present various Machine Learning methods. Here we focus mainly on the functionality of \textbf{scikit-learn}.
\bpycod
# Common imports
import numpy as np
import pandas as pd
import matplotlib.pyplot as plt
import sklearn.linear_model as skl
from sklearn.model_selection import train_test_split
from sklearn.metrics import mean_squared_error, r2_score, mean_absolute_error
import os

# Where to save the figures and data files
PROJECT_ROOT_DIR = "Results"
FIGURE_ID = "Results/FigureFiles"
DATA_ID = "DataFiles/"

if not os.path.exists(PROJECT_ROOT_DIR):
    os.mkdir(PROJECT_ROOT_DIR)

if not os.path.exists(FIGURE_ID):
    os.makedirs(FIGURE_ID)

if not os.path.exists(DATA_ID):
    os.makedirs(DATA_ID)

def image_path(fig_id):
    return os.path.join(FIGURE_ID, fig_id)

def data_path(dat_id):
    return os.path.join(DATA_ID, dat_id)

def save_fig(fig_id):
    plt.savefig(image_path(fig_id) + ".png", format='png')

infile = open(data_path("MassEval2016.dat"),'r')
\epycod


Before we proceed, we define also a function for making our plots. You can obviously avoid this and simply set up various \textbf{matplotlib} commands every time you need them. You may however find it convenient to collect all such commands in one function and simply call this function. 
\bpycod
from pylab import plt, mpl
plt.style.use('seaborn')
mpl.rcParams['font.family'] = 'serif'

def MakePlot(x,y, styles, labels, axlabels):
    plt.figure(figsize=(10,6))
    for i in range(len(x)):
        plt.plot(x[i], y[i], styles[i], label = labels[i])
        plt.xlabel(axlabels[0])
        plt.ylabel(axlabels[1])
    plt.legend(loc=0)
\epycod

Our next step is to read the data on experimental binding energies and
reorganize them as functions of the mass number $A$, the number of
protons $Z$ and neutrons $N$ using \textbf{pandas}.  Before we do this it is
always useful (unless you have a binary file or other types of compressed
data) to actually open the file and simply take a look at it!


In particular, the program that outputs the final nuclear masses is written in Fortran with a specific format. It means that we need to figure out the format and which columns contain the data we are interested in. Pandas comes with a function that reads formatted output. After having admired the file, we are now ready to start massaging it with \textbf{pandas}. The file begins with some basic format information.
\bpycod
"""                                                                                                                         
This is taken from the data file of the mass 2016 evaluation.                                                               
All files are 3436 lines long with 124 character per line.                                                                  
       Headers are 39 lines long.                                                                                           
   col 1     :  Fortran character control: 1 = page feed  0 = line feed                                                     
   format    :  a1,i3,i5,i5,i5,1x,a3,a4,1x,f13.5,f11.5,f11.3,f9.3,1x,a2,f11.3,f9.3,1x,i3,1x,f12.5,f11.5                     
   These formats are reflected in the pandas widths variable below, see the statement                                       
   widths=(1,3,5,5,5,1,3,4,1,13,11,11,9,1,2,11,9,1,3,1,12,11,1),                                                            
   Pandas has also a variable header, with length 39 in this case.                                                          
"""
\epycod

The data we are interested in are in columns 2, 3, 4 and 11, giving us
the number of neutrons, protons, mass numbers and binding energies,
respectively. We add also for the sake of completeness the element name. The data are in fixed-width formatted lines and we will
covert them into the \textbf{pandas} DataFrame structure.

\bpycod
# Read the experimental data with Pandas
Masses = pd.read_fwf(infile, usecols=(2,3,4,6,11),
              names=('N', 'Z', 'A', 'Element', 'Ebinding'),
              widths=(1,3,5,5,5,1,3,4,1,13,11,11,9,1,2,11,9,1,3,1,12,11,1),
              header=39,
              index_col=False)

# Extrapolated values are indicated by '#' in place of the decimal place, so
# the Ebinding column won't be numeric. Coerce to float and drop these entries.
Masses['Ebinding'] = pd.to_numeric(Masses['Ebinding'], errors='coerce')
Masses = Masses.dropna()
# Convert from keV to MeV.
Masses['Ebinding'] /= 1000

# Group the DataFrame by nucleon number, A.
Masses = Masses.groupby('A')
# Find the rows of the grouped DataFrame with the maximum binding energy.
Masses = Masses.apply(lambda t: t[t.Ebinding==t.Ebinding.max()])
\epycod

We have now read in the data, grouped them according to the variables we are interested in. 
We see how easy it is to reorganize the data using \textbf{pandas}. If we
were to do these operations in C/C++ or Fortran, we would have had to
write various functions/subroutines which perform the above
reorganizations for us.  Having reorganized the data, we can now start
to make some simple fits using both the functionalities in \textbf{numpy} and
\textbf{scikitlearn} afterwards. Our first attempt is to make a fit using the
standard least squares functionality of \textbf{numpy}. Here we will also
show functionalities like the computation of the mean values, the
variance and the covariance. 

Now we define five variables which contain
the nucleon number, the number of protons and neutrons, the element name and finally the energies themselves.
\bpycod
A = Masses['A']
Z = Masses['Z']
N = Masses['N']
Element = Masses['Element']
Energies = Masses['Ebinding']
print(Masses)
\epycod
The next step, and we will define this mathematically later, is to set up the so-called \textbf{design matrix}. We will throughout call this matrix $\bm{X}$.
It has dimensionality $p\times n$, where $n$ is the number of data points and $p$ are the so-called predictors. In our case here they are given by the number of polynomials in $A$ we wish to include in the fit. 
\bpycod
# Now we set up the design matrix X
X = np.zeros((5,len(A)))
X[4,:] = A**(-1.0)
X[3,:] = A**(-1.0/3.0)
X[2,:] = A**(2.0/3.0)
X[1,:] = A
X[0,:] = 1
\epycod
With \textbf{scikitlearn} we are now ready to use linear regression and fit our data.
\bpycod
clf = skl.LinearRegression().fit(X.T, Energies)
fity = clf.predict(X.T)
\epycod
Pretty simple!  
Now we can print measures of how our fit is doing, the coefficients from the fits and plot the final fit together with our data.
\bpycod
# The mean squared error                               
print("Mean squared error: %.2f" % mean_squared_error(Energies, fity))
# Explained variance score: 1 is perfect prediction                                 
print('Variance score: %.2f' % r2_score(Energies, fity))
# Mean absolute error                                                           
print('Mean absolute error: %.2f' % mean_absolute_error(Energies, fity))
print(clf.coef_, clf.intercept_)

Masses['Eapprox']  = fity
# Generate a plot comparing the experimental with the fitted values values.
fig, ax = plt.subplots()
ax.set_xlabel(r'$A = N + Z$')
ax.set_ylabel(r'$E_\mathrm{bind}\,/\mathrm{MeV}$')
ax.plot(Masses['A'], Masses['Ebinding'], alpha=0.7, lw=2,
            label='Ame2016')
ax.plot(Masses['A'], Masses['Eapprox'], alpha=0.7, lw=2, c='m',
            label='Fit')
ax.legend()
save_fig("Masses2016")
plt.show()
\epycod


\paragraph{Seeing the wood for the trees.}
As a teaser, let us now see how we can do this with decision trees using \textbf{scikit-learn}. Thereafter we switch to \textbf{random forests}!


\bpycod

#Decision Tree Regression
from sklearn.tree import DecisionTreeRegressor
regr_1=DecisionTreeRegressor(max_depth=5)
regr_2=DecisionTreeRegressor(max_depth=7)
regr_3=DecisionTreeRegressor(max_depth=9)
regr_1.fit(X.T, Energies)
regr_2.fit(X.T, Energies)
regr_3.fit(X.T, Energies)


y_1 = regr_1.predict(X.T)
y_2 = regr_2.predict(X.T)
y_3=regr_3.predict(X.T)
Masses['Eapprox'] = y_3
# Plot the results
plt.figure()
plt.plot(A, Energies, color="blue", label="Data", linewidth=2)
plt.plot(A, y_1, color="red", label="max_depth=5", linewidth=2)
plt.plot(A, y_2, color="green", label="max_depth=7", linewidth=2)
plt.plot(A, y_3, color="m", label="max_depth=9", linewidth=2)

plt.xlabel("$A$")
plt.ylabel("$E$[MeV]")
plt.title("Decision Tree Regression")
plt.legend()
save_fig("Masses2016Trees")
plt.show()
print(Masses)
print(np.mean( (Energies-y_1)**2))
\epycod


\paragraph{And what about using neural networks?}
The \textbf{seaborn} package allows us to visualize data in an efficient way. Note that we use \textbf{scikit-learn}'s multi-layer perceptron (or feed forward neural network) 
functionality.
\bpycod
from sklearn.neural_network import MLPRegressor
from sklearn.metrics import accuracy_score
import seaborn as sns

X_train = X.T
Y_train = Energies
n_hidden_neurons = 100
epochs = 100
# store models for later use
eta_vals = np.logspace(-5, 1, 7)
lmbd_vals = np.logspace(-5, 1, 7)
# store the models for later use
DNN_scikit = np.zeros((len(eta_vals), len(lmbd_vals)), dtype=object)
train_accuracy = np.zeros((len(eta_vals), len(lmbd_vals)))
sns.set()
for i, eta in enumerate(eta_vals):
    for j, lmbd in enumerate(lmbd_vals):
        dnn = MLPRegressor(hidden_layer_sizes=(n_hidden_neurons), activation='logistic',
                            alpha=lmbd, learning_rate_init=eta, max_iter=epochs)
        dnn.fit(X_train, Y_train)
        DNN_scikit[i][j] = dnn
        train_accuracy[i][j] = dnn.score(X_train, Y_train)

fig, ax = plt.subplots(figsize = (10, 10))
sns.heatmap(train_accuracy, annot=True, ax=ax, cmap="viridis")
ax.set_title("Training Accuracy")
ax.set_ylabel("$\eta$")
ax.set_xlabel("$\lambda$")
plt.show()



\epycod



\paragraph{More on flexibility with pandas and xarray.}
Let us study the $Q$ values associated with the removal of one or two nucleons from
a nucleus. These are conventionally defined in terms of the one-nucleon and two-nucleon
separation energies. With the functionality in \textbf{pandas}, two to three lines of code will allow us to plot the separation energies.
The neutron separation energy is defined as 

\[
S_n= -Q_n= BE(N,Z)-BE(N-1,Z),
\]
and the proton separation energy reads
\[
S_p= -Q_p= BE(N,Z)-BE(N,Z-1).
\]
The two-neutron separation energy is defined as
\[
S_{2n}= -Q_{2n}= BE(N,Z)-BE(N-2,Z),
\]
and  the two-proton separation energy is given by
\[
S_{2p}= -Q_{2p}= BE(N,Z)-BE(N,Z-2).
\]

Using say the neutron separation energies (alternatively the proton separation energies)
\[
S_n= -Q_n= BE(N,Z)-BE(N-1,Z),
\]
we can define the so-called energy gap for neutrons (or protons) as 
\[
\Delta S_n= BE(N,Z)-BE(N-1,Z)-\left(BE(N+1,Z)-BE(N,Z)\right),
\]
or 
\[
\Delta S_n= 2BE(N,Z)-BE(N-1,Z)-BE(N+1,Z).
\]
This quantity can in turn be used to determine which nuclei are magic or not. 
For protons we would have 
\[
\Delta S_p= 2BE(N,Z)-BE(N,Z-1)-BE(N,Z+1).
\]

To calculate say the neutron separation we need to multiply or masses with the nucleon number $A$ (why?).
The we pick the oxygen isotopes and simply compute the separation energis in two lines of code. 
\bpycod
# Regression analysis using scikit-learn functions
# Common imports
import numpy as np
import pandas as pd
import matplotlib.pyplot as plt
import os
from pylab import plt, mpl
plt.style.use('seaborn')
mpl.rcParams['font.family'] = 'serif'

def MakePlot(x,y, styles, labels, axlabels):
    plt.figure(figsize=(10,6))
    for i in range(len(x)):
        plt.plot(x[i], y[i], styles[i], label = labels[i])
        plt.xlabel(axlabels[0])
        plt.ylabel(axlabels[1])
    plt.legend(loc=0)



# Where to save the figures and data files
PROJECT_ROOT_DIR = "Results"
FIGURE_ID = "Results/FigureFiles"
DATA_ID = "DataFiles/"

if not os.path.exists(PROJECT_ROOT_DIR):
    os.mkdir(PROJECT_ROOT_DIR)

if not os.path.exists(FIGURE_ID):
    os.makedirs(FIGURE_ID)

if not os.path.exists(DATA_ID):
    os.makedirs(DATA_ID)

def image_path(fig_id):
    return os.path.join(FIGURE_ID, fig_id)

def data_path(dat_id):
    return os.path.join(DATA_ID, dat_id)

def save_fig(fig_id):
    plt.savefig(image_path(fig_id) + ".png", format='png')

infile = open(data_path("MassEval2016.dat"),'r')


# Read the experimental data with Pandas
Masses = pd.read_fwf(infile, usecols=(2,3,4,6,11),
              names=('N', 'Z', 'A', 'Element', 'Ebinding'),
              widths=(1,3,5,5,5,1,3,4,1,13,11,11,9,1,2,11,9,1,3,1,12,11,1),
              header=39,
              index_col=False)

# Extrapolated values are indicated by '#' in place of the decimal place, so
# the Ebinding column won't be numeric. Coerce to float and drop these entries.
Masses['Ebinding'] = pd.to_numeric(Masses['Ebinding'], errors='coerce')
Masses = Masses.dropna()
# Convert from keV to MeV.
Masses['Ebinding'] /= 1000
A = Masses['A']
Z = Masses['Z']
N = Masses['N']
Element = Masses['Element']
Energies = Masses['Ebinding']*A

df = pd.DataFrame({'A':A,'Z':Z, 'N':N,'Element':Element,'Energies':Energies})
Nucleus = df.loc[lambda df: df.Z==8, :]
# drop cases with no number
Nucleus = Nucleus.dropna()
print(Nucleus)
Nucleus['NeutronSeparationEnergies'] = Nucleus['Energies'].diff(+1)
MakePlot([Nucleus.A], [Nucleus.NeutronSeparationEnergies], ['b'], ['Neutron Separation Energy'], ['$A$','$S_n$'])
save_fig('Nucleus')
plt.show()

\epycod






\subsection{A first summary}

The aim of these introductory words was to present to you various
Python packages and their functionalities, in particular packages like
\textbf{numpy}, \textbf{pandas}, \textbf{xarray} and \textbf{matplotlib} make our life much easier
in handling various data sets and vsualizing data. 

Furthermore,
\textbf{scikit-learn} allows us with few lines of code to implement popular
Machine Learning algorithms.


\subsection{Exercises}

% ------------------- end of main content ---------------

% #ifdef PREAMBLE
\end{document}
% #endif

